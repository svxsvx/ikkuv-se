\documentclass[12pt,a4paper]{beamer}
%\usepackage[utf8x]{inputenc}
%\usepackage{ucs}
\usepackage{amsmath}
\usepackage{amsfonts}
\usepackage{amssymb}
\usepackage{graphicx}
\usepackage{wrapfig}
\usepackage{verbatim}

\author{Robin Ellerkmann, Sven Reber}
\title{Konflikthandhabung\\Theorie und Praxis}

\begin{document}
\maketitle

\begin{frame}
	\frametitle{Inhalt}
	
	\begin{itemize}
		\item Aufgabenstellung
		\item Fehlerfindung
		\begin{itemize}
			\item Verfahren 1: Nachbarwerte
			\item Verfahren 2: Quadratischer Fehler
			\item Verfahren 3: Einfach
		\end{itemize}
		\item Programmvorf\"uhrung
		\item Code-Rosinen
	\end{itemize}
\end{frame}

\begin{frame}
	\frametitle{Modelle der Konflikthandhabung}
	\begin{itemize}
		\item Modellieren Konfliktverhalten zwischen zwei Parteien
	\end{itemize}
\end{frame}

\begin{frame}
	\frametitle{Prozessmodell}
	\begin{itemize}
		\item Beschreibt interne Dynamik eines Konflikt als Ablauf von Phasen
		\item Events identifizieren und deren Bedeutung für weitere Events ermitteln
		\item Dyadische Konflikte laufen in Eventzyklen ab
	\end{itemize}
\end{frame}

\begin{frame}
	\frametitle{Abbildung Prozessmodell}
	\begin{itemize}
%//TODO: Add image from screenshot
	\end{itemize}
\end{frame}

\begin{frame}
	\frametitle{Frustration}
	\begin{itemize}
		\item Ausgangspunkt für Konflikte: Frustration bei einer der Parteien
		\item Frustration kann viele Formen haben
	\end{itemize}
\end{frame}

\begin{frame}
%//TODO: Adjust name
	\frametitle{Wahrnehmung}
	\begin{itemize}
		\item Subjektive Definition des Anliegens für beide Parteien
		\item Drei Dimensionen bestimmen diese Definition:
		\begin{itemize}
			\item Egozentrik
			\item Einblick in zugrundeliegende Belange
			\item Größe/Wichtigkeit des Problems
		\end{itemize}
		\item Bewusstsein über mögliche Handlungen und deren Folgen ist begrenzt
	\end{itemize}
\end{frame}

\begin{frame}
	\frametitle{Verhalten}
	\begin{itemize}
		\item Besteht aus drei Komponenten: Orientierung, strategische Ziele und Taktiken
		\item Orientierung: Wie wichtig ist einer Partei die Erfüllung des eigenen Anliegens? Wie wichtig ist die Erfüllung des Anliegens der anderen Partei?
		\item Strategische Ziele: Anpassung der Verhaltensweisen an den Gegenüber
		\item Taktiken: Beschreiben bestimmte Verhaltensweisen der Parteien. Z.B als Wettbewerbstaktik, Kooperative Taktik, Bargaining Taktik
	\end{itemize}
\end{frame}

\begin{frame}
	\frametitle{Interaktion}
	\begin{itemize}
		\item Zwei Perspektiven: Verhaltensweisen sind selbst gewählt oder Verhaltensweisen werden durch Aktionen der anderen Partei ausgelöst
		\item Selbst gewähltes Verhalten: Veränderung der Wahrnehmung des Konflikts ändert Verhalten
		\item Ausgelöstes Verhalten: Resultiert aus psychologischen Dynamiken, z.B. Eskalation/Deeskalation
	\end{itemize}
\end{frame}

\begin{frame}
	\frametitle{Ergebnis}
	\begin{itemize}
		\item Nachwirkungen des Konflikts
		\item Langzeiteffekte 
	\end{itemize}
\end{frame}

\begin{frame}
	\frametitle{Strukturmodell}
	\begin{itemize}
		\item Parameter identifizieren, die das Konfliktverhalten der Parteien beeinflussen 
	\end{itemize}
\end{frame}

%//Komponenten einfügen

\begin{frame}
	\frametitle{Vergleich der Modelle}
	\begin{itemize}
		\item 
	\end{itemize}
\end{frame}

%//Optional
\begin{frame}
	\frametitle{Zusammenfassung der Modelle}
	\begin{itemize}
		\item 
	\end{itemize}
\end{frame}

%//Führungsstile
\begin{frame}
	\frametitle{5-Punkt Modell Übersicht}
	\begin{itemize}
		\item 
	\end{itemize}
\end{frame}

%//Folien zu einzelnen Führungsstilen
%//Withdrawing, inaction: Robin
%//Others: Sven

%//Jeder remote sein Zeug
%//Fazit Folie

\end{document}
