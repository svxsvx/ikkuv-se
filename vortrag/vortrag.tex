\documentclass[12pt,a4paper]{beamer}
%\usepackage[utf8x]{inputenc}
%\usepackage{ucs}
\usepackage[ansinew]{inputenc} % Kodierung
\usepackage{ngerman} % Sprache
\usepackage{amsmath}
\usepackage{amsfonts}
\usepackage{amssymb}
\usepackage{graphicx}
\usepackage{wrapfig}
\usepackage{verbatim}
\usepackage{xcolor}

\title{Konflikthandhabung\\Theorie und Praxis}
\author{Robin Ellerkmann, Sven Reber}
\date{4. Februar 2017}

\begin{document}
\maketitle

\begin{frame}
	\frametitle{Inhalt}
	
	% FIXME: am Ende noch mal dem "echten" Inhalt/Titeln anpassen.
	\begin{itemize}
		\item Einleitung und Definition eines Konfliktes		% Sven
		\item Vergleich der Modelle					% Robin
		\begin{itemize}
			\item Prozessmodell
			\item Strukturmodell
		\end{itemize}
		\item F\"uhrungsstile							% Sven (ersten 3), Robin (widh-drawing, inaction)
		% XXX: F�r den Inahlt vllt. doch etwas zu viel/fr�h. "Erschl�gt" vllt. etwas. (Sven)		
		%\begin{itemize}
		%	\item Vermeidung (in-action)				% Robin
		%	\item Machteinsatz						% Sven
		%	\item Kompromiss						% Sven
		%	\item Anpassung (with-drawing)				% Robin
		%	\item Zusammenarbeit (problem solving)		% Sven
		%\end{itemize}
		\item Fazit									% Robin (?)
	\end{itemize}
\end{frame}

\begin{frame}
	\frametitle{Modelle der Konflikthandhabung}
	\begin{itemize}
		\item Modellieren Konfliktverhalten zwischen zwei Parteien
		\item Zwei Ans\"atze:
		\begin{itemize}
			\item Prozessmodell
			\item Strukturmodell
		\end{itemize}
	\end{itemize}
\end{frame}

\begin{frame}
	\frametitle{Prozessmodell}
	\begin{itemize}
		\item Beschreibt interne Dynamik eines Konfliktes als Ablauf von Phasen
		\item Events identifizieren und deren Bedeutung f\"ur weitere Events ermitteln
		\item Dyadische Konflikte laufen in Eventzyklen ab
	\end{itemize}
\end{frame}

\begin{frame}
	\frametitle{Abbildung Prozessmodell}
	\begin{figure}[p]
		\includegraphics[width=0.4\textwidth]{images/process_model.png}
	\end{figure}
\end{frame}

\begin{frame}
	\frametitle{Frustration}
	\begin{itemize}
		\item Ausgangspunkt f\"ur Konflikte: Frustration bei einer der Parteien
		\item Frustration kann viele Formen haben
	\end{itemize}
\end{frame}

\begin{frame}
%//TODO: Adjust name
	\frametitle{Wahrnehmung}
	\begin{itemize}
		\item Subjektive Definition des Anliegens f\"ur beide Parteien
		\item Drei Dimensionen bestimmen diese Definition:
		\begin{itemize}
			\item Egozentrik
			\item Einblick in zugrundeliegende Belange
			%\item Gr\"oße/Wichtigkeit des Problems
			\item Gr��e\,/\,Wichtigkeit des Problems
		\end{itemize}
		\item Bewusstsein \"uber m\"ogliche Handlungen und deren Folgen ist begrenzt
	\end{itemize}
\end{frame}

\begin{frame}
	\frametitle{Verhalten}
	\begin{itemize}
		\item Besteht aus drei Komponenten: Orientierung, strategische Ziele und Taktiken
		\item Orientierung: Wie wichtig ist einer Partei die Er\"ullung des eigenen Anliegens? Wie wichtig ist die Er\"ullung des Anliegens der anderen Partei?
		\item Strategische Ziele: Anpassung der Verhaltensweisen an den Gegen\"uber
		\item Taktiken: Beschreiben bestimmte Verhaltensweisen der Parteien. Z.\,B. als Wettbewerbstaktik, Kooperative Taktik, Bargaining Taktik
	\end{itemize}
\end{frame}

\begin{frame}
	\frametitle{Interaktion}
	\begin{itemize}
		\item Zwei Perspektiven: Verhaltensweisen sind selbst gew�hlt oder Verhaltensweisen werden durch Aktionen der anderen Partei ausgel\"ost
		\item Selbst gew�hltes Verhalten: Ver�nderung der Wahrnehmung des Konflikts �ndert Verhalten
		\item Ausgel\"ostes Verhalten: Resultiert aus psychologischen Dynamiken, z.\,B. Eskalation\,/\,Deeskalation
	\end{itemize}
\end{frame}

\begin{frame}
	\frametitle{Ergebnis}
	\begin{itemize}
		\item Nachwirkungen des Konflikts
		\item Langzeiteffekte 
	\end{itemize}
\end{frame}

\begin{frame}
	\frametitle{Strukturmodell}
	\begin{itemize}
		\item Identifiziert Parameter, die das Konfliktverhalten der Parteien beeinflussen
		\item Drei Arten von Einfluss auf das Konfliktverhalten jeder Partei:
		\begin{itemize}
			\item Verhaltensabsichten der Partei
			\item Sozialer Druck auf die Partei
			\item Beziehung zwischen den Interessen der Parteien
		\end{itemize}
	\end{itemize}
\end{frame}

\begin{frame}
	\frametitle{Abbildung Strukturmodell}
	\begin{figure}[p]
 		\includegraphics[width=1.0\textwidth]{images/structural_model.png}
	\end{figure}
\end{frame}

\begin{frame}
	\frametitle{Verhaltensabsichten}
	\begin{itemize}
		\item Dominanter Stil: Prim\"arziel, dass erreicht werden soll
		\item Backup Stil: Falls das Prim\"arziel nicht erreicht werden kann werden alternative Ziele verfolgt
	\end{itemize}
\end{frame}

\begin{frame}
	\frametitle{Sozialer Druck}
	\begin{itemize}
		\item Druck des Auftraggebers\,/\,der repr\"asentierten Gruppe
		\item Umgebender sozialer Druck durch neutrale Beobachter oder kulturelle Werte 
	\end{itemize}
\end{frame}

\begin{frame}
	\frametitle{Beziehung zwischen den Interessen der Parteien}
	\begin{itemize}
		\item Abh\"angig davon, ob ein Interessenkonflikt vorliegt
		\begin{itemize}
			\item Wettbewerb: Knappe Ressourcen erm\"oglichen nur die Umsetzung der Interessen einer Partei
			\item Gemeinsame Probleme: F\"ordert kooperatives Verhalten
			\item Kombination aus beiden
		\end{itemize}
	\end{itemize}
\end{frame}

%//Komponenten einf�gen

%//Optional
\begin{frame}
	\frametitle{Zusammenfassung der Modelle}
	\begin{itemize}
		\item Prozessmodell
		\begin{itemize}
			\item Beschreibt interne Dynamik eines Konfliktes als Ablauf von Phasen
			\item Phasen bilden abh\"angige Eventzyklen
		\end{itemize}
		\item Strukturmodell
		\begin{itemize}
			\item Beschreibt intern einen Konflikt als Mischung von Druck und Interessen von verschiedenen Parteien
		\end{itemize}
	\end{itemize}
\end{frame}

%//F�hrungsstile
\begin{frame}
	\frametitle{5-Punkt Modell \"Ubersicht}		
	\begin{itemize}
		\item Vermeidung \textcolor{lightgray}{(in-action)}			% Robin
		\item Machteinsatz \textcolor{lightgray}{(contending)}		% Sven
		\item Kompromiss \textcolor{lightgray}{(compromising)}		% Sven
		\item Anpassung \textcolor{lightgray}{(with-drawing)}		% Robin
		\item Zusammenarbeit \textcolor{lightgray}{(problem solving)}	% Sven
	\end{itemize}
\end{frame}

\begin{frame}
	\frametitle{Vermeidung}
	\begin{itemize}
		\item Konflikte werden ignoriert
		\item Bei Konfrontation: Flucht
		\item Unterscheidung von kurz- und langfristiger Vermeidung
	\end{itemize}
\end{frame}

\begin{frame}
	\frametitle{Anpassung}
	\begin{itemize}
		\item Erf\"ullung der W\"unsche der anderen Partei ohne R\"ucksicht auf eigene Interessen
		\item Kann verschiedene Gr\"unde haben
		\begin{itemize}
			\item Akzeptanz der (fachlichen) \"Uberlegenheit der anderen Partei
			\item Aufbau von sozialem Kredit
			\item Gesichtswahrung bei Hinzuziehen eines Mediators
		\end{itemize}
	\end{itemize}
\end{frame}

%//Folien zu einzelnen F�hrungsstilen
%//Withdrawing, inaction: Robin
%//Others: Sven

%//Jeder remote sein Zeug
%//Fazit Folie

\end{document}
