\documentclass[12pt,a4paper]{beamer}
%\usepackage[utf8x]{inputenc}
%\usepackage{ucs}
\usepackage{amsmath}
\usepackage{amsfonts}
\usepackage{amssymb}
\usepackage{graphicx}
\usepackage{wrapfig}
\usepackage{verbatim}

\author{Robin Ellerkmann, Sven Reber}
\title{Konflikthandhabung\\Theorie und Praxis}

\begin{document}
\maketitle

\begin{frame}
	\frametitle{Inhalt}
	
	\begin{itemize}
		\item Aufgabenstellung
		\item Fehlerfindung
		\begin{itemize}
			\item Verfahren 1: Nachbarwerte
			\item Verfahren 2: Quadratischer Fehler
			\item Verfahren 3: Einfach
		\end{itemize}
		\item Programmvorf\"uhrung
		\item Code-Rosinen
	\end{itemize}
\end{frame}

\begin{frame}
	\frametitle{Modelle der Konflikthandhabung}
	\begin{itemize}
		\item Modellieren Konfliktverhalten zwischen zwei Parteien
	\end{itemize}
\end{frame}

\begin{frame}
	\frametitle{Prozessmodell}
	\begin{itemize}
		\item Beschreibt interne Dynamik eines Konflikt als Ablauf von Phasen
		\item Events identifizieren und deren Bedeutung für weitere Events ermitteln
	\end{itemize}
\end{frame}

%//PHASEN einfügen

\begin{frame}
	\frametitle{Strukturmodell}
	\begin{itemize}
		\item Parameter identifizieren, die das Konfliktverhalten der Parteien beeinflussen 
	\end{itemize}
\end{frame}

%//Komponenten einfügen

\begin{frame}
	\frametitle{Vergleich der Modelle}
	\begin{itemize}
		\item 
	\end{itemize}
\end{frame}

%//Optional
\begin{frame}
	\frametitle{Zusammenfassung der Modelle}
	\begin{itemize}
		\item 
	\end{itemize}
\end{frame}

\end{document}
